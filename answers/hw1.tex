\documentclass[11pt]{article}
\usepackage[UTF8]{ctex}
\usepackage[a4paper]{geometry}
\geometry{left=2.0cm,right=2.0cm,top=2.5cm,bottom=2.5cm}

\usepackage{comment}
\usepackage{booktabs}
\usepackage{graphicx}
\usepackage{diagbox}
\usepackage{amsmath,amsfonts,graphicx,amssymb,bm,amsthm}
%\usepackage{algorithm,algorithmicx}
\usepackage[ruled]{algorithm2e}
\usepackage[noend]{algpseudocode}
\usepackage{fancyhdr}
\usepackage{tikz}
\usepackage{graphicx}
\usetikzlibrary{arrows,automata}
\usepackage{hyperref}
\hypersetup{
	colorlinks=true,
	linkcolor=blue,
	filecolor=blue,      
	urlcolor=blue,
	citecolor=cyan,
}			

\setlength{\headheight}{14pt}
\setlength{\parindent}{0 in}
\setlength{\parskip}{0.5 em}

\newtheorem{theorem}{Theorem}
\newtheorem{lemma}[theorem]{Lemma}
\newtheorem{proposition}[theorem]{Proposition}
\newtheorem{claim}[theorem]{Claim}
\newtheorem{corollary}[theorem]{Corollary}
\newtheorem{definition}[theorem]{Definition}
\newtheorem*{definition*}{Definition}

\newenvironment{problem}[2][Problem]{\begin{trivlist}
\item[\hskip \labelsep {\bfseries #1}\hskip \labelsep {\bfseries #2.}]}{\hfill$\blacktriangleleft$\end{trivlist}}
\newenvironment{answer}[1][Answer]{\begin{trivlist}
\item[\hskip \labelsep {\bfseries #1.}\hskip \labelsep]}{\hfill$\lhd$\end{trivlist}}

\newcommand\E{\mathbb{E}}
\newcommand\per{\mathrm{per}}


\title{Homework \#1}
\usetikzlibrary{positioning}

\begin{document}

\pagestyle{fancy}
\lhead{Peking University}
\chead{}
\rhead{Mathematical Foundations for the Information Age, 2024 Fall}

\begin{center}
    {\LARGE \bf Homework \#1}\\
    {Due: 2024-9-30 23:59 \quad$|$\quad 8 Problems, 100 Pts}\\
    {Name: 徐靖, ID: 2200012917}            % Write down your name and ID here.
\end{center}



\begin{problem}{1 (8')}
Define
\begin{align*}
    \Gamma(s) = \int_{0}^{+\infty} x^{s - 1} \mathrm{e}^{-x} \mathrm{d}x.
\end{align*}
where $s>0$. It can be proved that $\Gamma(s)$ is well-defined (You don't need to prove this).
\begin{itemize}
    \item [(1)] (4') Prove that, $\Gamma(s+1)=s\Gamma(s)$.
    \item [(2)] (4') Prove that,
    \begin{align*}
        \Gamma(s)=2\int_{0}^{+\infty}x^{2s-1}\mathrm{e}^{-x^2}\mathrm{d}x.
    \end{align*}
\end{itemize}
\end{problem}

\begin{answer} ~
\begin{itemize}
    \item [(1)] Enter your answer here.
    \item [(2)] Enter your answer here.
\end{itemize}
\end{answer}



\begin{problem}{2 (10')}
Define a random variable $Q \sim \chi^2(k)$ $(k\in\mathbb{Z}_+)$ if $Q=Z_1^2+\cdots+Z_k^2$ where $Z_1,\cdots,Z_k \sim \mathcal{N}(0,1)$ are independent random variables. Given two independent random variables $X\sim\chi^2(m),\;Y\sim\chi^2(n)$ $(m,n\in\mathbb{Z}_+,\;m > n)$.
\begin{itemize}
    \item [(1)] (4') Calculate the value of $\mathbb{E}(X)$.
    \item [(2)] (2') Prove that, $X + Y \sim \chi^2(m + n)$.
    \item [(3)] (4') Does $X - Y \sim \chi^2(m - n)$? Prove your result.
\end{itemize}
\end{problem}

\begin{answer} ~
\begin{itemize}
    \item [(1)] Enter your answer here.
    \item [(2)] Enter your answer here.
    \item [(3)] Enter your answer here. 
\end{itemize}
\end{answer}



\begin{problem}{3 (8')} In this problem, we will prove two basic probability inequalities. You will use these inequalities frequently in this course, so make sure you are familiar with them, including the statements and conditions.
\begin{itemize}
    \item [(1)] (4') (Markov Inequality) Suppose $X$ is a non-negative random variable and $\mathbb{E}(X)<+\infty$. Prove that, for any $a>0$,
    \begin{align*}
        \mathbb{P}(X\geq a)\leq\frac{\mathbb{E}(X)}{a}.
    \end{align*}
    \item [(2)] (4') (Chebyshev Inequality) Suppose $X$ is a random variable with $\mathbb{E}(X)<+\infty, Var(X)<+\infty$. Prove that, for any $a>0$,
    \begin{align*}
        \mathbb{P}(|X-\mathbb{E}(X)|\geq a)\leq\frac{Var(X)}{a^2}.
    \end{align*}
\end{itemize}
\end{problem}

\begin{answer} ~
\begin{itemize}
    \item [(1)] Enter your answer here.
    \item [(2)] Enter your answer here.
\end{itemize}
\end{answer}



\begin{problem}{4 (14')}
In this problem, we will prove the following Chernoff bound (Theorem \ref*{thm:chernoff}).
\begin{theorem}[Chernoff Bound]
\label{thm:chernoff}
    Suppose $X_1,\cdots,X_n$ are independently Bernoulli random variables with expectation $p\in(0,1)$. Then for any $\varepsilon\in(0,1-p)$,
    \begin{align*}
        \mathbb{P}\left(\frac{1}{n}\sum_{i=1}^n X_i\geq p+\varepsilon\right)\leq \exp\left[-nD_B(p+\varepsilon||p)\right] \leq \exp(-2n\varepsilon^2),
    \end{align*} 
where 
\begin{align*}
    D_B(p||q):=p\ln\frac{p}{q}+(1-p)\ln\frac{1-p}{1-q}.
\end{align*}
\end{theorem}
\begin{itemize}
    \item [(1)] (5') Under the conditions of Theorem \ref*{thm:chernoff}, prove that for any $t>0$,
    \begin{align*}
        \mathbb{P}\left(\sum_{i=1}^n X_i\geq n(p+\varepsilon)\right)\leq\mathbb{E}\left(\mathrm{e}^{t\sum_{i=1}^n X_i}\right)\cdot\mathrm{e}^{-nt(p+\varepsilon)}.
    \end{align*}
    \item [(2)] (9') Finish the proof of Theorem \ref*{thm:chernoff}.
\end{itemize}
\end{problem}

\begin{answer} ~
\begin{itemize}
    \item [(1)] Enter your answer here.
    \item [(2)] Enter your answer here.
\end{itemize}
\end{answer}



\begin{problem}{5 (12')} ~
\begin{itemize}
    \item [(1)] (3') For what value of $d$ does the volume $V(d)$ of a $d$-dimensional unit ball take on its maximum? You don't need to prove your result.
    \item [(2)] (4') Consider drawing a random point $x$ from the unit ball in $\mathbb{R}^d$ (surface and interior) uniformly at random. What's the variance of $x_1$ (the first coordinate of $x$)? You don't need to prove your result.
    \item [(3)] (5') Recall the way we generate a random unit vector in $d$ dimensions. First, generate $d$ i.i.d samples $v_i\;(i\in [d])$ from a Gaussian distribution with $\mu=0$ and $\sigma^2=1$. Then, define the vector $v$ as $v=[v_1,v_2,...,v_d]$. Finally, return $\frac{v}{||v||_2}$, where $||v||_2=\sqrt{\sum_{i=1}^d v_i^2}$. Briefly explain why it generates a unit vector uniformly at random.
\end{itemize}
\end{problem}

\begin{answer} ~
\begin{itemize}
    \item [(1)] Enter your answer here.
    \item [(2)] Enter your answer here.
    \item [(3)] Enter your answer here. 
\end{itemize}
\end{answer}



\begin{problem}{6 (12')}
A 3-dimensional cube has vertices, edges and faces. In a $d$-dimensional cube, these components are called faces. A vertex is a $0$-dimensional face, an edge a $1$-dimensional face, etc. Answer the following problems. You don't need to prove your result.
\begin{itemize}
    \item [(1)] (3') For $0\leq k\leq d$, how many $k$-dimensional faces does a $d$-dimensional cube have?
    \item [(2)] (3') What is the total number of faces of all dimensions? The $d$-dimensional face is the cube itself which you can include in your count.
    \item [(3)] (3') What is the surface area of a unit cube in $d$-dimensions (a unit cube has side-length one in each dimension)?
    \item [(4)] (3') What is the surface area of the cube if the length of each side was 2?
\end{itemize}
\end{problem}

\begin{answer} ~
\begin{itemize}
    \item [(1)] Enter your answer here.
    \item [(2)] Enter your answer here.
    \item [(3)] Enter your answer here. 
    \item [(4)] Enter your answer here.
\end{itemize}
\end{answer}



\begin{problem}{7 (16')}
Consider the upper hemisphere of a unit-radius ball in $d$-dimensions. What is the height of the maximum volume cylinder that can be placed entirely inside the hemisphere?

\textit{[Hint: You need to consider all possible placement options.]}
\end{problem}

\begin{answer}
Enter your answer here.
\end{answer}



\begin{problem}{8 (20')} 
Consider the following geometries in $d$-dimensional space:
\begin{align*}
    \Phi_d=\left\{(x_1,\cdots,x_d)\mid x_1^2+\cdots+x_d^2\leq 1\right\}\qquad
    \Omega_d=\left\{(x_1,\cdots,x_d)\mid |x_1|+\cdots+|x_d|\leq 1\right\}
\end{align*}
\begin{itemize}
    \item [(1)] (6') Consider the following two random processes:
    \begin{itemize}
        \item Pick two uniformly random unit vectors $x,y$ from the sphere in $d$ dimension.
        \item Pick a uniformly random plane passing through the origin in $d$ dimensions. Then pick two uniformly random unit vectors $w,z$ from the 2D sphere lying on that plane.
    \end{itemize}
    Determine whether $x$ and $w$ are identically distributed, and whether the pairs $(x,y)$ and $(w,z)$ are identically distributed. Briefly explain the reasons.
    \item [(2)] (6') Suppose $x$ and $y$ are sampled independently from $\Phi_d$. Denote random variable $Z=x^\top y$. Calculate $Var(Z)$.
    \item [(3)] (8') Calculate $V(\Omega_d)$ (the volume of $\Omega_d$) and $A(\Omega_d)$ (the surface area of $\Omega_d$).
\end{itemize}
\end{problem}

\begin{answer} ~
\begin{itemize}
    \item [(1)] Enter your answer here.
    \item [(2)] Enter your answer here.
    \item [(3)] Enter your answer here. 
\end{itemize}
\end{answer}



\end{document}