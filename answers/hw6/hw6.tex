\documentclass[11pt]{article}
\usepackage[UTF8]{ctex}
\usepackage[a4paper]{geometry}
\geometry{left=2.0cm,right=2.0cm,top=2.5cm,bottom=2.5cm}

\usepackage{comment}
\usepackage{booktabs}
\usepackage{graphicx}
\usepackage{diagbox}
\usepackage{amsmath,amsfonts,graphicx,amssymb,bm,amsthm}
%\usepackage{algorithm,algorithmicx}
\usepackage[ruled]{algorithm2e}
\usepackage[noend]{algpseudocode}
\usepackage{fancyhdr}
\usepackage{tikz}
\usepackage{graphicx}
\usetikzlibrary{arrows,automata}
\usepackage{hyperref}
\hypersetup{
	colorlinks=true,
	linkcolor=blue,
	filecolor=blue,      
	urlcolor=blue,
	citecolor=cyan,
}			

\setlength{\headheight}{14pt}
\setlength{\parindent}{0 in}
\setlength{\parskip}{0.5 em}

\newtheorem{theorem}{Theorem}
\newtheorem{lemma}[theorem]{Lemma}
\newtheorem{proposition}[theorem]{Proposition}
\newtheorem{claim}[theorem]{Claim}
\newtheorem{corollary}[theorem]{Corollary}
\newtheorem{definition}[theorem]{Definition}
\newtheorem*{definition*}{Definition}

\newenvironment{problem}[2][Problem]{\begin{trivlist}
\item[\hskip \labelsep {\bfseries #1}\hskip \labelsep {\bfseries #2.}]}{\hfill$\blacktriangleleft$\end{trivlist}}
\newenvironment{answer}[1][Answer]{\begin{trivlist}
\item[\hskip \labelsep {\bfseries #1.}\hskip \labelsep]}{\hfill$\lhd$\end{trivlist}}

\newcommand\E{\mathbb{E}}
\newcommand\per{\mathrm{per}}

\renewcommand{\figurename}{Figure}

\title{Homework \#6}
\usetikzlibrary{positioning}

\begin{document}

\pagestyle{fancy}
\lhead{Peking University}
\chead{}
\rhead{Mathematical Foundations for the Information Age, 2024 Fall}

\begin{center}
    {\LARGE \bf Homework \#6}\\
    {Due: 2025-1-9 23:59 \quad$|$\quad 3 Problems, 50 Pts}\\
    {Name: 徐靖, ID: 2200012917}            % Write down your name and ID here.
\end{center}



\begin{problem}{1 (14')} 
Consider the following algorithm to estimate the frequency of any number in data streams. The numbers in the data stream are in $[n]:=\{1,2,\cdots,n\}$. 

\begin{algorithm}[htbp]
    \caption{Estimate the frequency of numbers in data streams}
    $\bm C\gets 0$ \hfill{$\triangleright$ $\bm C$ is a $t\times k$-dimension matrix}

    Choose $t$ independent hash functions $h_1,\cdots,h_t:[n]\to[k]$ from a $2$-universal hash family

    \For{$j$ in data streams}{
        \For{$i=1,2,\cdots,t$}{
            $\bm C[i][h_i(j)]\gets \bm C[i][h_i(j)]+1$
        }
    }

    \For{$a$ in queries}{
        \textbf{output} $\hat f(a)=\min_{1\leq i\leq t}\bm C[i][h_i(a)]$
    }
\end{algorithm}

For any element $a$, suppose the real frequency of $a$ is $f(a)$. When $k=\left\lceil\frac{2}{\epsilon}\right\rceil,\;t=\left\lceil\log_2 \frac{1}{\delta}\right\rceil$, prove that for any given $a$, with probability at least $1-\delta$, $f(a)\leq\hat f(a)\leq f(a)+\epsilon L$, where $L$ is the length of the data stream. 
\end{problem}
\begin{answer}
    Let $X_i = \frac{C[i][h_i(a)]-f(a)}{L}$ and $A_i$ denotes $X_i>\epsilon$. Obviously we have,
$$C[i][h_i(a)]=\sum_{b\in\{b|h_i(a)=h_i(b)\}}f(b)\ge f(a)$$
Thus we know $X_i\ge 0$ and $P(\hat f(a)<f(a)) = P(\min_{i\in [t]} X_i<0)=0$.

    Given that $\forall x,y\in[k],a\neq b\in [n],P(h_i(a)=x,h_i(b)=y)=\frac{1}{k}$, we could get:
    $$\begin{align*}\sum_{i=1}^tC[i][h_i(a)]-f(a)&=\sum_{i=1}^t\left(-f(a)+\sum_{b\in\{b|h_i(a)=h_i(b)\}}f(b)\right)\\&=\sum_{(i,b)\in\{(x,y)|y\neq a,h_{x}(a)=h_x(b)\}}f(b)\\&=\sum_{b\in [n]\setminus\{a\}}tP(h_i(a)=h_i(b))f(b)\\&=\frac{t}{k}(L-f(a))\end{align*}$$
    which means $E(X_i) = \frac{L-f(a)}{kL}\leq \frac{1}{k}$. Using Markow inequality, we have:
    $$P(A_i)=P(X_i>\epsilon)\leq \frac{E(X_i)}{\epsilon}\leq \frac{1}{\epsilon k}$$
    Then,
    $$P(\hat f(a)>f(a)+\epsilon)=P(\min_{i\in [t]}X_i>\epsilon)=P(\cap A_i) = P(A_i)^t\leq \left(\frac{1}{\epsilon k}\right)^t\leq \delta$$
    In conclusion, we have $P(f(a)\leq\hat f(a)\leq f(a)+\epsilon L)>1-\delta$.
    
\end{answer}

\begin{problem}{2 (18')}
A bipartite graph is a graph whose vertices can be divided into two disjoint and independent sets $U$ and $V$ such that every edge connects a vertex in $U$ to one in $V$. Find out and prove the threshold for $\mathcal{G}(n,p)$ to be bipartite.

\textit{[Hint: The definition of bipartite graph is equivalent to a graph that does not contain any odd-length cycles.]}
\end{problem}

\begin{answer}
    We know bipartite graph is equivalent to a graph that does not contain any odd-length cycles. So let $X_k$ denotes the number of k-length cycles in the graph, and $X_S=\sum_{k\in S}X_k$, and index set $I=\{3,5,7,\dots,2[\frac{n+1}{2}]\}$. Given that,
$$P(G \text{ contain odd-length cycles})=P(X_I>0)=\sum_{k\in \mathbb N}P(X_I=k)\leq\sum_{k\in \mathbb N}kP(X_I=k)=E(X_I)$$
    Let \( S \) be the set of all places in the graph where a cycle could occur. Explicitly, \( S_k \) is the set of all subsets of \( k \) vertices ordered up to rotation and orientation of the cycle, and \( S = \bigcup_{k \in I} S_k \).

For \( s \in S \), define \( A_s \) to be the event that a odd-length cycle occurs on \( s \) in the random graph . As expectation is linear, we have:
\[
E(X_I) = \sum_{s\in S} E(1_{A_S}) = \sum_{k \in I} \sum_{s \in S_k} P(A_s).
\]

For \( s \in S_k \), the probability that a cycle occurs on \( s \) is \( p^k \), as we need each of the \( k \) independent edges which form the cycle to be present in our random graph. We want to determine \( |S_k| \). The number of ordered sets of size \( k \) is \( \binom{n}{k} k! \), which overcounts each \( S \in S_k \) by \( 2k \) times. 

Hence, 
\[
|S_k| = \frac{\binom{n}{k} k!}{2^k} = \binom{n}{k} \frac{(k-1)!}{2}.
\]

Thus, by the above formula:
\[
E(X_I) = \sum_{k \in I} \binom{n}{k} \frac{(k-1)!}{2} p^k.
\]

Now, note that \( \binom{n}{i} i! = n(n-1)\dots(n-i+1) \leq n^i \), and we get:
\[
E(X_I) \leq \sum_{k \in I} n^kp^k \leq \frac{n^3 p^3}{1 - n^2p^2}.
\]

Thus, as \( E(X_I) \to 0 \) for \( np \to 0 \), we have \( P(G \text{ has a odd-length cycle}) \leq E(X_I) \), and we have proven that, with high probability , \( G \) is bipartite.

For the second part of the proof, we have $0\leq X_3\leq X_I$.As demonstrated in class, $$P(X_I=0)\leq P(X_3=0)\leq \frac{\text{Var}(X_3)}{E^2(X_3)}\sim\frac{n^3p^3+n^4p^5}{n^6p^6}\to0(n\to 0)$$ 
Hence $\frac{1}{n}$ is the threshold.
\end{answer}

\begin{problem}{3 (18')}
A vertex is called an isolated vertex if it does not have any edges. Prove that, the threshold for $\mathcal{G}(n,p)$ of the existence of isolated vertex is $p=\frac{\ln n}{n}$.
\end{problem}

\begin{answer}
Let \( E_i \) be the event that vertex \( v_i \) is isolated in \( G_{n,p} \), and let \( E \) be the event that at least one vertex is isolated in \( G_{n,p} \). I have proved a stronger conclusion: In \( G_{n,p} \) with \( p = \frac{\ln n + c}{n} \), the probability that there is an isolated vertex converges to \( 1 - e^{-e^{-c}} \). 

First, I want to compute \( P(E) = P\left( \bigcup_{i=1}^{n} E_i \right) \). By the inclusion-exclusion principle, we have:

\[
P(E) = \sum_{k=1}^{n} (-1)^k \sum_{1 \leq i_1 < i_2 < \cdots < i_k \leq n} P\left( \bigcap_{j=1}^{k} E_{i_j} \right).
\]

Using Bonferroni inequalities, we get:

\[
P(E) \leq \sum_{k=1}^{l} (-1)^k \sum_{1 \leq i_1 < \cdots < i_k \leq n} P\left( \bigcap_{j=1}^{k} E_{i_j} \right), \quad \text{for odd} \ l.
\]

Now, calculate \( P\left( \bigcap_{j=1}^{k} E_{i_j} \right) \), which is the probability that the set of \( k \) vertices \( v_{i_1}, v_{i_2}, \dots, v_{i_k} \) are all isolated:

\[
P\left( \bigcap_{j=1}^{k} E_{i_j} \right) = (1 - p)^{n-k} \cdot \binom{k}{2} = (1 - p)^{n-k} \cdot \frac{k(k-1)}{2}.
\]

Thus, we have:

\[
P(E) \leq \sum_{k=1}^{l} (-1)^k \binom{n}{k} (1 - p)^{n-k} \cdot \frac{k(k-1)}{2}, \quad \text{for odd} \ l.
\]

As \( n \to \infty \), we have:

\[
\binom{n}{k} (1 - p)^{n-k} \cdot \frac{k(k-1)}{2} \sim \frac{n^k}{k!} \cdot e^{-ck}.
\]

Therefore, the summation becomes:

\[
P(E) \sim \sum_{k=1}^{\infty} (-1)^k \frac{e^{-ck}}{k!}.
\]

For odd \( l \), we have:

\[
\lim_{n \to \infty} P(E) \leq \sum_{k=1}^{\infty} (-1)^k \frac{e^{-c}}{k!} = 1 - e^{-e^{-c}}.
\]

Similarly, for even \( l \), we get:

\[
\lim_{n \to \infty} P(E) \geq \sum_{k=1}^{\infty} (-1)^k \frac{e^{-c}}{k!} = 1 - e^{-e^{-c}}.
\]

Altogether, we conclude that:

\[
\lim_{n \to \infty} P(E) = 1 - e^{-e^{-c}}.
\]

Back to the original question, we find that when $c\to +\infty$, $P(E)=1-e^{-e^{-c}}\to 0$, or the probability of an isolated point appearing is 0. And when $c\to +\infty$, $P(E)\to 1$

So the threshold is $\frac{\ln n}{n}$

\end{answer}

\end{document}